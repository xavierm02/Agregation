\providecommand{\base}{../..}
\documentclass[../../agregation.tex]{subfiles}
\begin{document}

\lec{907}{\todo Algorithmique du texte : exemples et applications.}

\subsection{Rapport du jury}

\begin{aquote}{\fullcite{agreg2015}}
Cette leçon devrait permettre au candidat de présenter une grande variété d'algorithmes et de paradigmes de programmation, et ne devrait pas se limiter au seul problème de la recherche d'un motif dans un texte, surtout si le candidat ne sait présenter que la méthode naïve.

De même, des structures de données plus riches que les tableaux de caractères peuvent montrer leur utilité dans certains algorithmes, qu'il s'agisse d'automates ou d'arbres par exemple.

Cependant, cette leçon ne doit pas être confondue avec la \emph{909 : Langages rationnels. Exemples et applications} ni avec la \emph{910 : Langages algébriques. Exemples et applications}.


La compression de texte peut faire partie de cette leçon si les algorithmes présentés contiennent effectivement des opérations comme les comparaisons de chaînes : la compression LZW, par exemple, ressortit davantage à cette leçon que la compression de Huffman.
\end{aquote}

\dvts

\end{document}