\providecommand{\base}{../..}
\documentclass[../../agregation.tex]{subfiles}
\begin{document}

\lec{928}{Problèmes NP-complets : exemples de réductions}

\subsection{Rapport du jury}

\begin{aquote}{\fullcite{agreg2015}}
L'objectif ne doit pas être de dresser un catalogue le plus exhaustif possible ; en revanche, pour chaque exemple, il est attendu que le candidat puisse au moins expliquer clairement le problème considéré, et indiquer de quel autre problème une réduction permet de prouver sa NP-complétude.

Les exemples de réduction seront autant que possible choisis dans des domaines variés : graphes, arithmétique, logique, etc. Un exemple de problème NP-complet dans sa généralité qui devient P si on contraint davantage les hypothèses pourra être présenté, ou encore un algorithme P approximant un problème NP-complet.

Si les dessins sont les bienvenus lors du développement, le jury attend une définition claire et concise de la fonction associant, à toute instance du premier problème, une instance du second ainsi que la preuve rigoureuse que cette fonction permet la réduction choisie.
\end{aquote}

\dvts

\end{document}