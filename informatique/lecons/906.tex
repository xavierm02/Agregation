\providecommand{\base}{../..}
\documentclass[../../agregation.tex]{subfiles}
\begin{document}

\lec{906}{Programmation dynamique : exemples et applications.}

\subsection{Rapport du jury}

\begin{aquote}{\fullcite{agreg2015}}
Même s'il s'agit d'une leçon d'exemples et d'applications, le jury attend des candidats qu'ils présentent les idées générales de la programmation dynamique et en particulier qu'ils aient compris le caractère générique de la technique de mémoïsation. Le jury appréciera que les exemples choisis par le candidat couvrent des domaines variés, et ne se limitent pas au calcul de la longueur de la plus grande sous-séquence commune à deux chaînes de caractères.


Le jury ne manquera pas d'interroger plus particulièrement le candidat sur la question de la correction des algorithmes proposés et sur la question de leur complexité en espace.
\end{aquote}

\subsection{Développements}

\dvts

\end{document}