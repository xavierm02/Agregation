\providecommand{\base}{..}
\documentclass[../agregation.tex]{subfiles}
\begin{document}

\part{Mathématiques}

\setcounter{tocdepth}{1}
\tableofcontents
\setcounter{tocdepth}{3}

\newrefsegment


\chapter{Leçons}

\lec{104}{Groupes finis. Exemples et applications.}

\subsection{Développements}

\dvts



\lec{105}{Groupe des permutations d'un ensemble fini. Applications.}

\subsection{Développements}

\dvts



\lec{106}{Groupe linéaire d'un espace vectoriel de dimension finie $E$, sous-groupes de $\operatorname{GL\left(E\right)}$. Applications.}

\subsection{Développements}

\dvts



\lec{108}{Exemples de parties génératrices d'un groupe. Applications.}

\subsection{Développements}

\dvts



\lec{120}{Anneaux $\mathbb Z / n\mathbb Z$. Applications.}

\subsection{Développements}

\dvts



\lec{121}{Nombres premiers. Applications.}

\subsection{Développements}

\dvts



\lec{123}{Corps finis. Applications.}

\subsection{Développements}

\dvts



\lec{141}{Polynômes irréductibles à une indéterminée. Corps de rupture. Exemples et applications.}

\subsection{Développements}

\dvts



\lec{150}{Exemples d'actions de groupes sur les espaces de matrices.}

\subsection{Développements}

\dvts



\lec{151}{Dimension d'un espace vectoriel (on se limitera au cas de la dimension finie). Rang. Exemples et applications.}

\subsection{Développements}

\dvts



\lec{152}{Déterminant. Exemples et applications.}

\subsection{Développements}

\dvts



\lec{153}{Polynômes d'endomorphisme en dimension finie. Réduction d'un endomorphisme en dimension finie. Applications.}

\subsection{Développements}

\dvts



\lec{157}{Endomorphismes trigonalisables. Endomorphismes nilpotents.}

\subsection{Développements}

\dvts



\lec{159}{Formes linéaires et hyperplans en dimension finie. Exemples et applications.}

\subsection{Développements}

\dvts



\lec{162}{Systèmes d'équations linéaires ; opérations, aspects algorithmiques et conséquences théoriques.}

\subsection{Développements}

\dvts



\lec{170}{Formes quadratiques sur un espace vectoriel de dimension finie. Orthogonalité, isotropie. Applications.}

\subsection{Développements}

\dvts



\lec{181}{Barycentres dans un espace affine réel de dimension finie, convexité. Applications.}

\subsection{Développements}

\dvts



\lec{182}{Applications des nombres complexes à la géométrie.}

\subsection{Développements}

\dvts



\lec{183}{Utilisation des groupes en géométrie.}

\subsection{Développements}

\dvts



\lec{190}{Méthodes combinatoires, problèmes de dénombrement.}

\subsection{Développements}

\dvts










\chapter{Développements}

\dvt{frobeniuszolotarev}{Théorème de Frobenius-Zolotarev}

Leçons :
\begin{itemize}
	\vadans{105}
	\vadans{106}
	\vadans{120}
	\vadans{121}
	\vadans{123}
	\vadans{152}
\end{itemize}

\dvt{sousgroupescompactsdeglnr}{Sous-groupes compacts de $\operatorname{GL}_n\left(\mathbb R\right)$}

Leçons :
\begin{itemize}
	\vadans{106}
	\vadans{150}
	\vadans[bof]{151}
	\vadans{181}
	\vadans[bof]{183}
\end{itemize}

\end{document}
