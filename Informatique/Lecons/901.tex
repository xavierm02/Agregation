\providecommand{\base}{../..}
\documentclass[../../Agregation.tex]{subfiles}
\begin{document}

\lec{901}{\todo Structures de données : exemples et applications.}

\subsection{Rapport du jury}

\begin{aquote}{\fullcite{agreg2015}}
Le mot \emph{algorithme} ne figure pas dans l'intitulé de cette leçon, même si l'utilisation des structures de données est évidemment fortement liée à des questions algorithmiques.

La leçon doit donc être orientée plutôt sur la question du choix d'une structure de données que d'un algorithme. Le jury attend du candidat qu'il présente différents types abstraits de structures de données en donnant quelques exemples de leur usage avant de s'intéresser au choix de la structure concrète. Le candidat ne peut se limiter à des structures linéaires simples comme des tableaux ou des listes, mais doit présenter également quelques structures plus complexes, reposant par exemple sur des implantations à l'aide d'arbres.
\end{aquote}

\end{document}