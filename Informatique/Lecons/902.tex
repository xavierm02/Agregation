\providecommand{\base}{../..}
\documentclass[../../Agregation.tex]{subfiles}
\begin{document}

\lec{902}{\todo Diviser pour régner : exemples et applications.}

\subsection{Rapport du jury}

\begin{aquote}{\fullcite{agreg2015}}
Cette leçon permet au candidat de proposer différents algorithmes utilisant le paradigme \emph{diviser pour régner}. Le jury attend du candidat que ces exemples soient variés et touchent des domaines différents.

Un calcul de complexité ne peut se limiter au cas où la taille du problème est une puissance exacte de 2, ni à une application directe d'un théorème très général recopié approximativement d'un ouvrage de la bibliothèque de l'agrégation.
\end{aquote}

\end{document}