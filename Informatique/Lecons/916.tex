\providecommand{\base}{../..}
\documentclass[../../Agregation.tex]{subfiles}
\begin{document}

\lec{916}{\todo Formules du calcul propositionnel : représentation, formes normales, satisfiabilité. Applications.}

\subsection{Rapport du jury}

\begin{aquote}{\fullcite{agreg2015}}
Le jury attend des candidats qu'ils abordent les questions de la complexité de la satisfiabilité.

Pour autant, les applications ne sauraient se réduire à la réduction de problèmes NP-complets à SAT.

Une partie significative du plan doit être consacrée à la représentation des formules et à leurs formes normales.
\end{aquote}

\dvts

\subsection{Références}

\begin{itemize}
	\item Huth, Ryan (tableau avec différentes représentations p. 361, ROBDD)
	\item Stern (Tseitin)
	\item Devisme, Lafourcade, Lévy (Tout le reste, en particulier DPLL)
\end{itemize}

\subsection{Plan}

\begin{itemize}
	\item Syntaxe (lecture unique, notation parenthésée, notation polonaise, arbre, circuit)
	\item Sémantique (compacité, représentations [table, formes normales, OBDD])
	\item Problèmes associés (SAT, Valie, DPLL, Tseitin)
\end{itemize}

\subsection{Remarques}

Aucunes.

\end{document}