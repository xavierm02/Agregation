\providecommand{\base}{../..}
\documentclass[../../Agregation.tex]{subfiles}
\begin{document}

\lec{915}{Classes de complexité : exemples.}

\subsection{Rapport du jury}

\begin{aquote}{\fullcite{agreg2015}}
Le jury attend que le candidat aborde à la fois la complexité en temps et en espace. Il faut naturellement exhiber des exemples de problèmes appartenant aux classes de complexité introduites, et montrer les relations d'inclusion existantes entre ces classes.

Le jury s'attend à ce que le caractère strict ou non de ces inclusions soit abordé, en particulier le candidat doit être capable de montrer la non-appartenance de certains problèmes à certaines classes.

Parler de décidabilité dans cette leçon serait hors sujet.
\end{aquote}

\dvts

\end{document}