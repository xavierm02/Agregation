\providecommand{\base}{../..}
\documentclass[../../Agregation.tex]{subfiles}
\begin{document}

\lec{920}{Réécriture et formes normales. Exemples.}

\subsection{Rapport du jury}

\begin{aquote}{\fullcite{agreg2015}}
Au-delà des propriétés standards (terminaison, confluence) des systèmes de réécriture, le jury attend notamment du candidat qu'il présente des exemples sur lesquels l'étude des formes normales est pertinente dans des domaines variés : calcul formel, logique, etc.

Un candidat ne doit pas s'étonner que le jury lui demande de calculer des paires critiques sur un exemple concret.

Lorsqu'un résultat classique comme le lemme de Newman est évoqué, le jury attend du candidat qu'il sache le démontrer.
\end{aquote}

\dvts

\end{document}