\providecommand{\base}{../..}
\documentclass[../../Agregation.tex]{subfiles}
\begin{document}

\lec{927}{\todo Exemples de preuve d'algorithme : correction, terminaison.}

\subsection{Rapport du jury}

\begin{aquote}{\fullcite{agreg2015}}
Le jury attend du candidat qu'il traite des exemples d'algorithmes récursifs et des exemples d'algorithmes itératifs.

En particulier, le candidat doit présenter des exemples mettant en évidence l'intérêt de la notion d'invariant pour la correction partielle et celle de variant pour la terminaison des segments itératifs.

Une formalisation comme la logique de Hoare pourra utilement être introduite dans cette leçon, à condition toutefois que le candidat en maîtrise le langage.
\end{aquote}

\end{document}