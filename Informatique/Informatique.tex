\providecommand{\base}{..}
\documentclass[../Agregation.tex]{subfiles}
\begin{document}

\part{Informatique}

\tableofcontents

\newrefsegment

\chapter{Leçons}

\subfile{\base/Informatique/Lecons/901.tex}
\subfile{\base/Informatique/Lecons/902.tex}
\subfile{\base/Informatique/Lecons/903.tex}
\subfile{\base/Informatique/Lecons/906.tex}
%\subfile{\base/Informatique/Lecons/907.tex}
%\subfile{\base/Informatique/Lecons/909.tex}
%\subfile{\base/Informatique/Lecons/910.tex}
\subfile{\base/Informatique/Lecons/912.tex}
\subfile{\base/Informatique/Lecons/913.tex}
%%\subfile{\base/Informatique/Lecons/914.tex}
%%\subfile{\base/Informatique/Lecons/915.tex}
\subfile{\base/Informatique/Lecons/916.tex}
\subfile{\base/Informatique/Lecons/917.tex}
%%\subfile{\base/Informatique/Lecons/918.tex}
%%\subfile{\base/Informatique/Lecons/919.tex}
%%\subfile{\base/Informatique/Lecons/920.tex}
\subfile{\base/Informatique/Lecons/921.tex}
\subfile{\base/Informatique/Lecons/922.tex}
%\subfile{\base/Informatique/Lecons/923.tex}
%%\subfile{\base/Informatique/Lecons/924.tex}
\subfile{\base/Informatique/Lecons/925.tex}
%%\subfile{\base/Informatique/Lecons/926.tex}
\subfile{\base/Informatique/Lecons/927.tex}
%%\subfile{\base/Informatique/Lecons/928.tex}





































\chapter{Développements}
\dvt{cook}{Théorème de Cook}
Leçons :
\begin{itemize}
	\vadans{913}
	\vadans{916}
\end{itemize}
Wolper (et Carton)

%\dvt{probindalg}{Problèmes indécidables sur les langages algébriques}
%Leçons :
%\begin{itemize}
%	\vadans{910}
%	\vadans{922}
%\end{itemize}

\dvt{tripartas}{Tri par Tas}
Leçons :
\begin{itemize}
	\vadans{901}
	\vadans{903}
	\vadans{927}
\end{itemize}
Cormen

Remarques :
\begin{itemize}
	\item $n$ pour taille du tas et $l$ pour longeueur de tableau
Faire un dessin.
	\item Dire que la série converge sans le justifier.
\end{itemize}

\dvt{unification}{Unification}
Leçons :
\begin{itemize}
	\vadans{917}
	\vadans{927}
\end{itemize}

\dvt{plssc}{Plus longue sous-séquence commune}
Leçons :
\begin{itemize}
	\vadans{906}
	%\vadans{907}
\end{itemize}

Cormen

Remarques :
\begin{itemize}
	\item Lemme, relation de récurrence et algorithme (et exemple si c'est trop court).
	\item Il faut juste changer les notations ($\widetilde{u}u_\bullet=u$)
\end{itemize}

%\dvt{plsscmemoire}{Plus longue sous-séquence commune (optimisation mémoire)}
%Leçons :
%\begin{itemize}
%	\vadans{902}
%\end{itemize}

\dvt{pointslesplusproches}{Les points les plus proches}
Leçons :
\begin{itemize}
	\vadans{902}
	\vadans{921}
\end{itemize}

%\dvt{voyageur}{Problème du voyageur de commerce euclidien}
%Leçons :
%\begin{itemize}
%	\vadans{}
%	\vadans{925}
%\end{itemize}

\dvt{bellmanford}{Bellman-Ford}
Leçons :
\begin{itemize}
	\vadans{906}
	\vadans{925}
\end{itemize}

Remarques :
\begin{itemize}
	\item Bien différentier taille et poids d'un chemin
	\item \emph{Le} poids minimal, \emph{un} chemin de poids minimal
\end{itemize}

\dvt{tritopologique}{Tri topologique}
Leçons :
\begin{itemize}
	\vadans{903}
	\vadans{925}
\end{itemize}
Cormen

Remarques :
\begin{itemize}
	\item Parcours en profondeur avec date
	\item intervalles de dates disjoint ou inclusion (on suppose u.d < v.d et après cas selon u.f < v.d ou non)
	\item si (u, v) arête alors u.f > v.f (cas selon la couleur de v quand on passe par (u, v))
	\item exemple ?
\end{itemize}


\dvt{abro}{Arbres binaires de recherche optimaux}
Leçons :
\begin{itemize}
	\vadans{901}
	\vadans{906}
	\vadans{921}
\end{itemize}

\dvt{presburger}{Décidabilité de l’arithmétique de Presburger}
Leçons :
\begin{itemize}
	%\vadans{909}
	\vadans{922}
\end{itemize}

%\dvt{trifusiontrirapide}{Tri fusion / Tri rapide}
%Leçons :
%\begin{itemize}
%	\vadans{902}
%	\vadans{903}
%\end{itemize}

\dvt{calculableimpliquerecursif}{Calculable implique récursif}
Leçons :
\begin{itemize}
	\vadans{912}
	\vadans{913}
\end{itemize}

Remarques :
\begin{itemize}
	\item Carton pour les formules
	\item Wolper four nom des fonctions intermédiaires et vision globale
	\item Il faut prendre $f~:~\Sigma^* \to \Gamma^*$ et à la fin, rajouter une fonction pour transformer $(u, q, v)$ et $\varepsilon, q, v')$ pour pouvoir dire que le résultat est $v'$.
\end{itemize}

\dvt{langagesre}{Caractérisation des langages RE}
Leçons :
\begin{itemize}
	\vadans{912}
	\vadans{922}
\end{itemize}


\dvt{compacité}{Théorème de compacité}
Leçons :
\begin{itemize}
	\vadans{916}
\end{itemize}

\dvt{fft}{FFT}
Leçons :
\begin{itemize}
	\vadans{901}
	\vadans{902}
\end{itemize}


\dvt{todo}{Lov-Sko scendant | prolog | completude de deduction naturelle}
Leçons :
\begin{itemize}
	\vadans{921}
\end{itemize}

\chapter{Références}

Livres a priori cools : \textcite{Kozen:1992:DAA:121450}

\printbibliography[segment=1,resetnumbers]

lassaigne rougemeont

froidevaux godel soria

cori lascar (1 ET 2)

dehornoy

informatique theorique (DEVISME, LAFOURCADE LEVY) : DPLL (et mieux pr calcul prop?)
\end{document}